\begin{frame}{Aprendizado de M�quina}
\begin{itemize}
\item As t�cnicas de \alert{Aprendizado de M�quina} t�m sido aplicadas com sucesso em um grande n�mero de problemas reais em diversos dom�nios.
\ \ \newline
\item Principal raz�o: natureza inferencial e a boa capacidade de generaliza��o dos m�todos e t�cnicas.
\ \ \newline
\item Algoritmos capazes de aprender padr�es por meio de exemplos, baseado-se em dados previamente dispon�veis.
\end{itemize}
\end{frame}

\begin{frame}{Vis�o Computacional}
\begin{itemize}
	\item A Vis�o Computacional �rea � um que procura desenvolver m�todos capazes de replicar nos computadores as capacidades da vis�o humana.
	\ \ \newline
	\item Procedimentos de extra��o de caracter�sticas de imagens envolvia um grande esfor�o devido. 
	\ \ \newline
	\item An�lises de especialistas eram utilizadas para descobrir as regras necess�rias para o reconhecimento e extra��o de certos padr�es.
\end{itemize}
\end{frame} 

\begin{frame}{Deep Learning}
\begin{itemize}
\item Necessidade de ferramentas e algoritmos mais sofisticados 
\begin{itemize}
	\item Crescente complexidade dos problemas a serem tratados computacionalmente
	\item Grande volume de dados constantemente gerados
\end{itemize}
\ \ \newline
\item Qt, permite desenvolver aplica��es e possui uma grande quantidade de ferramentas e bibliotecas para implementa��o de interfaces gr�ficas \citep{Nuno:Qt}
\ \ \newline
\item PyQt, combina as vantagens oferecidas pela linguagem e pelo software, permitindo todas as funcionalidades providas pelo Qt por meio da linguagem Python \citep{Harwani:PyQt}
\ \ \newline
\item Oferece todos os recursos gr�ficos e de intera��o necess�rios para uma aplicac�o
\end{itemize}
\end{frame} 

