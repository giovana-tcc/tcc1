\begin{frame}{Aprendizado de M�quina}
	\begin{itemize}
	\item As t�cnicas de \alert{Aprendizado de M�quina} t�m sido aplicadas com sucesso em um grande n�mero de problemas reais em diversos dom�nios
	\bigskip
	\pause
	\item Caracter�sticas: natureza inferencial e a boa capacidade de generaliza��o dos m�todos e t�cnicas desta �rea
	\bigskip
	\item Algoritmos capazes de aprender padr�es por meio de exemplos, baseado-se em dados previamente dispon�veis
	\end{itemize}
\end{frame}

\begin{frame}{Vis�o Computacional}
	\begin{itemize}
		\item A \alert{Vis�o Computacional} � uma �rea que procura desenvolver m�todos capazes de replicar nos computadores as capacidades da vis�o humana.
		\pause
		\bigskip
		\item Procedimentos de extra��o de caracter�sticas de imagens envolviam um grande esfor�o devido ao processamento
		\bigskip
		\item An�lises de especialistas eram utilizadas para descobrir as regras necess�rias para o reconhecimento e extra��o de certos padr�es
	\end{itemize}
\end{frame} 

\begin{frame}{Deep Learning}
	\begin{itemize}
	\item Algoritmos capazes de lidar com a demanda de c�lculos matem�ticos complexos automatizados
	\bigskip
	\item Necessidade de ferramentas e algoritmos mais sofisticados 
	\begin{itemize}
		\item Complexidade de problemas computacionais
		\item Grande volume de dados constantemente gerados
	\end{itemize}
	\bigskip
	\pause
	\item T�nicas de \alert{Deep Learning}: conjunto de camadas de redes neurais artificias 
	\begin{itemize}
		\item Simplicidade da constru��o das redes
		\item Escalabilidade
		\item Transfer�ncia de conhecimento
	\end{itemize}
	\end{itemize}
\end{frame} 

