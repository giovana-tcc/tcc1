\begin{frame}{Metodologia}
	\begin{itemize}
		\item[] A condu��o das atividades obedece � metodologia a seguir, composta dos seguintes passos:
		\medskip
		\begin{itemize}
			\footnotesize
			\item[1.] Estudo dos conceitos relacionados � Aprendizado de M�quina, \textit{Deep Learning} e as principais arquiteturas de redes neurais convolucionais; 
			\medskip
			\item[2.] Estudo do ferramental tecnol�gico para elabora��o e execu��o de projetos de \textit{Deep Learning}, incluindo Python, Keras, Sci-kit Learn, Google Cloud Platform, dentre outros;
			\medskip
			\item[3.] Elaborar uma base de dados representativa de imagens coloridas e em escalas de cinza para fins de aprendizado dos padr�es de colora��o pelas redes neurais convolucionais;
			\medskip
			\item[4.] Elencar um conjunto de arquiteturas can�nicas das redes neurais convolucionais aplic�veis ao problema em quest�o;
		\end{itemize}
	\end{itemize}
\end{frame}

\begin{frame}{Metodologia}
\begin{itemize}
	\item[]
	\begin{itemize}
		\footnotesize
		\item[5.] Propor modifica��es nas redes neurais identificadas no passo anterior mediante \textit{Transfer Learning};
		\medskip
		\item[6.] Treinar as redes modificadas com os exemplos da base de dados;
		\medskip
		\item[7.] Testar as redes e coletar m�tricas de desempenho;
		\medskip
		\item[8.] Analisar os resultados obtidos identificando as redes mais adequadas ao cen�rio considerado;
		\medskip
		\item[9.] Escrita da proposta do Trabalho de Conclus�o de Curso;
		\medskip
		\item[10.] Defesa da proposta do Trabalho de Conclus�o de Curso;
		\medskip
		\item[11.] Escrita do Trabalho de Conclus�o de Curso;
		\medskip
		\item[12.] Defesa do Trabalho de Conclus�o de Curso.
	\end{itemize}
\end{itemize}
\end{frame}