\emph{Transfer Learning} (TL), ou Transfer�ncia de Conhecimento, � uma poderosa t�cnica de DL a qual possui diversas aplica��es em diferentes dom�nios \cite{ref:gulli}. Ao inv�s de estruturar uma arquitetura de uma CNN e trein�-la por completo, esta t�cnica permite reutilizar uma rede pr�-treinada e adapt�-la a um novo conjunto de dados \cite{ref:sewak}. Modelos que foram pr�-treinados utilizando um vasto e gen�rico conjunto de dados conseguem capturar caracter�sticas universais, como por exemplo curvas e arestas, em suas primeiras camadas \cite{ref:zaccone}.

As t�cnicas de TL podem ser utilizadas de diferentes maneiras, baseando-se nas arquiteturas das CNNs. Existem alguns modelos dispon�veis para aplica��es que foram pr�-treinados utilizando as principais arquiteturas can�nicas de CNN e aprenderam as caracter�sticas de grandes conjuntos de dados bastante conhecidos, como o ImageNet e o Places205 \cite{ref:image-net,ref:places205}. Para diferentes tarefas, esses modelos podem ser alterados modificando a camada de sa�da e fazendo um retreinamento nas �ltimas camadas das redes para se obter o aprendizado desejado \cite{ref:khan}. 