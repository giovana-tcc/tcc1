As tecnologias e bibliotecas predominantemente utilizadas neste trabalho envolvem e s�o compat�veis com a linguagem de programa��o Python, pois esta tem se destacado amplamente em projetos de AM em diversos cen�rios. Esta � uma linguagem de programa��o interpretada, interativa, multi-paradigma,  de alto n�vel, multi-plataforma, com uma sintaxe simples e c�digo aberto, idealizada por Guido van Rossum no in�cio da d�cada de 1990 \cite{ref:python}.

Para o processamento das imagens em termos de redimensionamento, persist�ncia, mudan�a e consulta do espa�o de cores, as bibliotecas \texttt{PIL} (Pillow) e \texttt{colormath} tiveram um papel protagonista \cite{lib:pillow,lib:colormath}. No tocante � manipula��o de arquivos, contemplando abertura, leitura e busca por extens�es similares, as bibliotecas \texttt{os} e \texttt{glob} foram utilizadas \cite{lib:os,lib:glob}. A manipula��o do conjunto de imagens e de suas respectivas representa��es matriciais ficou por conta da \texttt{numpy}, uma biblioteca fundamental para computa��o cient�fica que � extremamente poderosa para gerenciamento e altera��es de matrizes de muitas dimens�es \cite{lib:numpy}.  Ademais, no que diz a respeito do treinamento e testes dos modelos de AM, as bibliotecas \texttt{scikit-learn} e \texttt{keras} foram consideradas, em que a primeira teve um papel principal nos c�lculos autom�ticos das m�tricas de desempenho e a segunda nos modelos de DL com par�metros previamente configurados \cite{lib:scikit,lib:keras}.

Por fim, a infra-estrutura de computa��o em nuvem provida pelo Google Cloud Platform (GCP) totalmente voltada para projetos de AM foi essencial para o desenvolvimento deste projeto. As m�quinas virtuais oferecidas pela plataforma aumentaram o poder computacional acess�vel possibilitando um processamento mais favor�vel ao cen�rio em quest�o. A manipula��o e pr�-processamento do conjunto �ntegro de imagens, o treinamento dos modelos de DL e a fase de an�lise desses modelos foram executados em inst�ncias dispon�veis pelo GCP tendo em vista os recursos de mem�ria e processamento necess�rios \cite{tec:gcloud}.
