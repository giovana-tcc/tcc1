Para atingir os objetivos propostos no escopo deste trabalho, a condu��o das atividades obedece � metodologia apresentada a seguir, composta dos seguintes passos:

\begin{enumerate}
	\item Estudo dos conceitos relacionados � Aprendizado de M�quina, \textit{Deep Learning} e as principais arquiteturas de redes neurais convolucionais;
	\item Estudo do ferramental tecnol�gico para elabora��o e execu��o de projetos de \textit{Deep Learning}, incluindo Python, Keras, Sci-kit Learn, Google Cloud Platform, dentre outros;
	\item Elaborar uma base de dados representativa de imagens coloridas e em escalas de cinza para fins de aprendizado dos padr�es de colora��o pelas redes neurais convolucionais;
	\item Elencar um conjunto de arquiteturas can�nicas das redes neurais convolucionais aplic�veis ao problema em quest�o;
	\item Propor modifica��es nas redes neurais identificadas no passo anterior mediante \textit{Transfer Learning};
	\item Treinar as redes modificadas com os exemplos da base de dados;
	\item Testar as redes e coletar m�tricas de desempenho;
	\item Analisar os resultados obtidos identificando as redes mais adequadas ao cen�rio considerado;
	\item Escrita da proposta do Trabalho de Conclus�o de Curso;
	\item Defesa da proposta do Trabalho de Conclus�o de Curso;
	\item Escrita do Trabalho de Conclus�o de Curso;
	\item Defesa do Trabalho de Conclus�o de Curso.
\end{enumerate}
