Todas as atividades inicialmente propostas na metodologia foram executadas conforme proposto no cronograma, sem descumprimento aos prazos inicialmente estabelecidos. Como resultado, destaca-se a descrição do problema segundo a perspectiva de uma tarefa de AM utilizando as técnicas de TL e DL.

Inicialmente, toda a base de dados foi consolidada e pré-processada conforme descrito na Seção \ref{subsec:pre-process} visando estruturar os dados para apresentá-los às CNNs. Este procedimento foi realizado utilizando a infra-estrutura de \emph{hardware} disponibilizada pela plataforma Google Cloud.

Preliminarmente, utilizando o método de validação cruzada \emph{holdout} com $70\%$ dos dados para treinamento, o modelo de arquitetura VGGNet16 pré-treinado com imagens do ImageNet foi a base para o TL realizado em um modelo treinado com uma pequena porcentagem do conjunto de dados para fins de testes. O treinamento foi executado em uma instância de máquina virtual da plataforma Google Cloud e obteve um tempo de execução de aproximadamente 6 minutos obtendo um MSE de $261,175$. O modelo foi treinado durante $3$ épocas, com uma taxa de aprendizado de $10^{-4}$ e todos os outros parâmetros preservados. Mais épocas e parâmetros serão considerados nas fases posteriores deste trabalho.
