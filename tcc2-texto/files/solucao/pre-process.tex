Como se sabe, algoritmos de AM precisam de quantidades significativas de dados que estejam preferencialmente sem muitos ruídos. Entretanto, com o aumento do tamanho do conjunto de dados os custos computacionais também aumentam e faz-se necessário um pré-processamento desses dados para estruturá-los da maneira ideal sem que haja uma excessiva sobrecarga computacional \cite{ref:marsland}.

O conjunto de imagens que será utilizado no processo de colorização deste trabalho será submetido a um pré-processamento que se inicia padronizando a dimensão das imagens. Primeiramente, as imagens são redimensionadas de forma que possuam os mesmos tamanhos de largura e altura, para obtenção de uma imagem quadrada. Em seguida, cada imagem é redimensionada para $128 \times 128$ pixels, padronizando todas as imagens nesta mesma dimensão. Esta padronização visa tornar a tarefa de aprendizado factível diante dos recursos computacionais disponíveis.

Após a etapa de redimensionamento, a etapa seguinte consistiu em selecionar as imagens cujo espaço de cores fosse o RGB. Isto foi efetuado com o intuito de eliminar imagens em tons de cinza e em outros espaços de cores que demandassem um maior processamento posterior. No caso das imagens em tons de cinza, em particular, a eliminação das mesmas simplifica o processo de treinamento dos modelos, visto que estas não fornecem informação relevantes para aprendizado de colorações. Nesta etapa, foram eliminadas cerca de $900$ imagens que se encaixam em um dos critérios mencionados.

Em seguida, todas as imagens foram então convertidas do espaço de cores RGB para o espaço de cores CIELab, tendo em vista a diminuição dos parâmetros de entrada e saída no processo de aprendizagem e também a adequação aos procedimentos descritos na Seção \ref{subsec:tarefa}. Cada imagem passou então a corresponder a uma matriz de dimensões $3\times 128 \times 128$, a qual foi separada em duas partes: componente $L$, de dimensões $128 \times 128$, a ser fornecida como \emph{entrada} para os modelos, e componentes $a$ e $b$, a serem fornecidos como rótulo de \emph{saída} para o aprendizado supervisionado dos modelos, com dimensões $2 \times 128 \times 128$.
