A proposta deste trabalho de conclusão de curso tem por objetivo endereçar a colorização de imagens. Neste problema, recebe-se como entrada uma imagem em tons de cinza e, a partir desta, obtém-se uma versão em cores de maneira automática e sem a intervenção humana direta. Uma colorização de boa qualidade é obtida quando tem-se uma correspondência razoável com o mundo real, sendo agradável ao olhar humano. Estas colorizações podem contribuir para recuperação de  imagens históricas, para melhor contextualização de imagens de câmeras de segurança, dentre outros, o que justifica  a importância  do problema considerado.

Levando em conta o que foi mencionado a respeito do problema, este trabalho de conclusão de curso se propõe a utilizar técnicas de Aprendizado de Máquina para endereçá-lo, em especial, utilizando as redes neurais convolucionais via \emph{Transfer Learning} segundo um problema de regressão de acordo com o paradigma supervisionado. Para realização do treinamento destas redes, a base de dados utilizada foi um subconjunto da ImageNet, cuja diversidade tende a contribuir para o cenário em questão. É interessante ressaltar que os dados passaram por um pré-processamento visando estruturar estes dados para a apresentação aos modelos de \emph{Deep Learning}.

O processo de aprendizagem utilizou a abordagem \emph{holdout} de validação cruzada para avaliar a capacidade de generalização dos modelos e a métrica de desempenho erro médio quadrático, típica na literatura para problemas de regressão. Os resultados parciais do treinamento e teste foram obtidos utilizando a arquitetura VGGNet e apenas $1\%$ do conjunto de dados devido à limitação de tempo de execução, resultando preliminarmente em um erro de $261,17$. Como mencionado, almeja-se a melhoria destes resultados nas etapas posteriores.

Seguindo o cronograma, a continuidade deste trabalho de conclusão de curso propõe a realização de treinamentos de outros modelos utilizando a base de dados completa e modelos pré-treinados das arquiteturas VGGNet e ResNet. Os testes para análise de desempenho serão realizados utilizando a métrica erro médio quadrático além de comparações entre os modelos treinados baseadas nos parâmetros utilizados.
