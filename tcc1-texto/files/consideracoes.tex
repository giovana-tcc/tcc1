\begin{comment}
A proposta deste trabalho de conclusão de curso apresenta os procedimentos necessários para endereçar o problema de colorização de imagens utilizando as técnicas de \emph{Deep Learning}. Este problema consiste em obter a coloração de imagens em tons de cinza por meio de transferência de conhecimento utilizando redes neurais convolucionais pré-treinadas e adaptadas a este cenário. 
	content...
\end{comment}

A proposta deste trabalho de conclusão de curso tem por objetivo endereçar a colorização de imagens. Neste problema, recebe-se como entrada uma imagem em tons de cinza e, a partir desta, obtém-se uma versão em cores, de maneira automática e sem a intervenção humana direta. Uma colorização de boa qualidade é obtida quando tem-se uma correspondência razoável com o mundo real, sendo agradável ao olhar humano. Estas colorizações podem contribuir para recuperação de  imagens históricas, para melhor contextualização de imagens de câmeras de segurança, dentre outros, o que justifica  a importância  do problema considerado.

Levando em conta o que foi mencionado a respeito do problema, este trabalho de conclusão de curso se propõe a utilizar técnicas de Aprendizado de Máquina para endereçá-lo, em especial, utilizando as  Redes neurais convolucionais via transfer learning segundo um problema regressão. (...)