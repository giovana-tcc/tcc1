Diversos trabalhos na literatura têm abordado o tema de colorização de imagens, como \cite{ref:rel-zhang,ref:rel-deshpande,ref:rel-welsh}. Esta seção apresenta as técnicas utilizadas por \cite{ref:rel-zhang} para o problema de colorização, as quais dispõem conceitos de ML e DL.

Zhang propõe uma abordagem totalmente automatizada que produz colorizações vibrantes e realistas utilizando CNNs \emph{feedfoward} treinadas com milhões de imagens. No trabalho em questão, o canal de luminosidade $L$ é apresentado a modelos que preveem os canais de cromaticidade $a$ e $b$ correspondentes de uma imagem no espaço de cores CIELab. No entanto, os autores ressaltam que este cenário produz imagens coloridas com baixa saturação. Considerando a colorização de um domínio multimodal, uma das explicações citadas para este problema são as métricas utilizadas que podem estimular previsões conservadoras.
 
Para contornar este problema, os autores adotaram algumas técnicas matemáticas que favorecem problemas multimodais realizando re-treinamentos para enfatizar as cores menos frequentemente utilizadas. Esta abordagem explora a diversidade do conjunto de dados e propõe resultados com cores mais vibrantes e perceptivamente mais realistas.

O principal objetivo do trabalho de Zhang era obter resultados visualmente agradáveis para o ser humano. Deste modo, os autores desenvolveram um método próprio de avaliação intitulado ``Colorization Turing Test'' (CTT). Este método apresentava aos avaliadores uma imagem colorida e a imagem correspondente colorida pelas redes neurais propostas e solicitava a identificação da imagem real. Os CTTs confundiram $32\%$ dos dos avaliadores, um resultado significativamente maior que outros trabalhos similares da literatura.

Alguns aspectos do trabalho de Zhang foram relevantes para o contexto deste trabalho. Um desses aspectos foi a estruturação dos atributos utilizados para o treinamento das CNNs. Considerando uma imagem no espaço de cores CIELab, a entrada dos modelos propostos será a luminosidade $L$ de uma dada imagem para obtenção das colorações $a$ e $b$ correspondentes. No entanto, devido a limitação de recursos computacionais, as técnicas utilizadas considerando o cenário multimodal não serão abordadas.

Sabendo que os recursos computacionais disponíveis são escassos, a solução encontrada para aprimorar o treinamento das redes neurais convolucionais foi a utilização das técnicas de TL, as quais utilizam modelos pré-treinados com grandes conjuntos de dados para extração de características. Essas técnicas disponibilizam os pesos ajustados durante o treinamento dos modelos como uma forma de transferência de conhecimento para agregar aprendizagem aos modelos adaptados ao cenário em questão.  
