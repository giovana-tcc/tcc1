Diversos trabalhos na literatura têm abordado o tema de colorização de imagens, considerando diferentes técnicas e abordagens, a exemplo de transferência matemática de cores e uso de florestas aleatórias  \cite{ref:rel-zhang,ref:rel-deshpande,ref:rel-welsh}. Dentre os trabalhos relacionados, destaca-se a contribuição de Zhang et al., que aproxima-se da abordagem considerada nesta proposta ao utilizar AM e DL.

Os autores propõe uma abordagem totalmente automatizada que produz colorizações vibrantes e realistas utilizando CNNs \emph{feedfoward} treinadas com milhões de imagens \cite{ref:rel-zhang}. No trabalho em questão, o canal de luminosidade $L$ é apresentado aos modelos que preveem os canais de cromaticidade $a$ e $b$ correspondentes de uma imagem no espaço de cores CIELab. No entanto, os autores ressaltam que este cenário produz imagens coloridas com baixa saturação. Uma justificativa para este problema é que várias colorações podem ser apropriadas para uma mesma entrada (por exemplo, atribuir a um carro as cores vermelho ou preto), o que não é exatamente capturado pelas métricas consideradas.

Para contornar o problema mencionado, os autores adotaram algumas técnicas matemáticas que favorecem problemas análogos, realizando re-treinamentos para enfatizar as cores menos frequentemente utilizadas. Esta abordagem explora a diversidade do conjunto de dados e propõe resultados com cores mais vibrantes e perceptivamente mais realistas \cite{ref:rel-zhang}.

%O principal objetivo do trabalho de Zhang et al. era obter resultados visualmente agradáveis para o ser humano. Deste modo, os autores desenvolveram um método próprio de avaliação intitulado ``Colorization Turing Test'' (CTT). Este método apresentava aos avaliadores uma imagem colorida e a imagem correspondente colorida pelas redes neurais propostas e solicitava a identificação da imagem real. Os CTTs confundiram $32\%$ dos dos avaliadores, um resultado significativamente maior que outros trabalhos similares da literatura.

Alguns aspectos do trabalho de Zhang et al. foram relevantes para o contexto deste trabalho. Um desses aspectos foi a estruturação dos atributos utilizados para o treinamento das CNNs. Considerando uma imagem no espaço de cores CIELab, a entrada dos modelos propostos será a luminosidade $L$ de uma dada imagem para obtenção das colorações $a$ e $b$ correspondentes. No entanto, devido a limitação de recursos computacionais e ao escopo de um trabalho de conclusão de curso, a ênfase matemática ou de técnicas não-triviais de pré e pós-processamento não serão consideradas neste momento. Apesar disso, considera-se o uso de um modelo de vanguarda em um problema desafiador.

Sabendo que os recursos computacionais disponíveis são limitados, a solução proposta para aprimorar o treinamento das redes neurais convolucionais foi a utilização das técnicas de TL, as quais utilizam modelos pré-treinados com grandes conjuntos de dados para extração de características. Essas técnicas disponibilizam os pesos ajustados durante o treinamento dos modelos como uma forma de transferência de conhecimento para agregar aprendizagem aos modelos adaptados ao cenário em questão.
