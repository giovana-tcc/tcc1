Algoritmos de ML precisam de quantidades significativas de dados que estejam preferencialmente sem muitos ruídos. Entretanto, com o aumento do tamanho do conjunto de dados os custos computacionais também aumentam e faz-se necessário um pré-processamento desses dados para estruturá-los da maneira ideal sem que haja uma excessiva sobrecarga computacional \cite{ref:marsland}.

O conjunto de imagens que será utilizado no processo de colorização será submetido a um pré-processamento que se inicia padronizando a dimensão das imagens \todo{inserir aqui que o redimensionamento se adequa ao problema de colorização e ao custo computacional}. Primeiramente, as imagens são redimensionadas de forma que possuam os mesmos tamanhos de largura e altura, para obtenção de uma imagem quadrada. Em seguida, a imagem é redimensionada em $128 \times 128$ pixels, padronizando todas as imagens nesta mesma dimensão.

Considerando as etapas de redimensionamento, a etapa seguinte consiste em filtrar o espaço de cores das imagens admitindo que o conjunto de dados possua somente imagens coloridas e que estejam no espaço de cores RGB. Em seguida, todas as imagens são convertidas de RGB para o espaço de cores CIELab tendo em vista a diminuição dos parâmetros de entrada e saída no processo de aprendizagem. Por fim, os valores de $L$, $a$ e $b$ de cada uma das imagens são registrados em uma matriz n-dimensional para posteriormente serem fornecidos aos modelos DL. 