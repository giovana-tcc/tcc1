Para o Passo 4 descrito na Seção \ref{subsec:tarefa} será considerada a abordagem de \emph{Transfer Learning}, visando aproveitar parâmetros de redes CNNs pré-treinadas com \emph{datasets} massivos, a exemplo do ImageNet. Para tanto, considerando a utilização do \emph{framework} Keras, foi consultada a biblioteca de modelos previamente treinados, disponíveis no módulo \texttt{applications}. Como resultado, foram identificadas as arquiteturas canônicas e suas características elencadas na Tabela \ref{tab:cnns}.

\begin{table}[!ht]
	\caption{Arquiteturas canônicas e suas características disponíveis no módulo \texttt{keras.applications}.}
	\centering
	\begin{tabular}{c c c c}
		\toprule
		 Arquitetura & Profundidade & Parâmetros & Tamanho \\
		\midrule
		Xception & 126 & $22,910,480$ & 88 MB \\
		VGG16 & 23 & $138,357,544$ & 528 MB \\
		VGG19 & 26 & $143,667,240$ & 549 MB \\
		ResNet50 & 168 & $25,636,712$ & 99 MB \\ 
		InceptionV3 & 159 & $23,851,784$ & 92 MB \\
		InceptionResNetV2 & 572 & $55,873,736$ & 215 MB \\
		MobileNet & 88 & $4,253,864$ & 17 MB \\
		\bottomrule
	\end{tabular}
	
	\label{tab:cnns}
\end{table}

É interessante notar que todos estes modelos disponíveis foram pré-treinados com a base de dados do ImageNet. Considerando a quantidade de imagens e a variedade desses exemplos, acredita-se que estes pesos podem contribuir positivamente para o problema da coloração dada a transferência de aprendizado.

Levando em conta os recursos computacionais disponíveis em nuvem para este projeto e o tempo de processamento, as CNNs VGG16, ResNet50 e InceptionV3 disponíveis no \texttt{keras.applications} serão ajustadas perante \emph{Transfer Learning} para a tarefa de coloração proposta.

Dessas redes pré-treinadas, as últimas camadas serão substituídas por camadas completamente conectadas com saída compatível com as matrizes $a$ e $b$ de coloração. Eventuais re-treinos serão efetuados para ajuste de interconexão dessa nova camada. Serão consideradas $100$ épocas para o treinamento dos modelos propostos e os demais parâmetros terão seus valores padrões preservados.

