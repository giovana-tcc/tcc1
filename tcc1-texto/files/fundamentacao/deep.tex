\emph{Deep Learning} (DL), ou Aprendizado Profundo, é uma subárea do AM e consiste em uma abordagem que utiliza uma grande quantidade de camadas de RNAs para aprender representações de um vasto volume dados \cite{ref:chollet,ref:khan}. Informalmente, a palavra \emph{profundo} se refere ao volume de camadas e neurônios presentes nas arquiteturas das redes \cite{ref:gulli}. Através de análise de padrões, os sistemas baseados em técnicas de DL são capazes de reconhecer, traduzir, sintetizar e até prever sinais das mais diferentes naturezas \cite{ref:JAI-2017}. Para tarefas voltadas para a área de Visão Computacional, os métodos utilizados pelas técnicas de DL envolvem as redes neurais chamadas Redes Neurais Convolucionais \cite{ref:khan}. 

As técnicas de DL têm sido aplicadas com êxito em diversos problemas de diferentes naturezas \cite{ref:JAI-2017}. Essas técnicas de vanguarda e a crescente disponibilidade de grandes conjuntos de dados de imagens médicas estão produzindo notáveis avanços na compreensão automatizada deste tipo de dado. Essa perspectiva propõe os chamados diagnósticos médicos auxiliados por computador, do inglês \emph{Computer-Aided Diagnosis} (CAD), os quais podem ser utilizados, por exemplo, para detecção automatizada de doenças e para segmentação de órgãos e lesões \cite{ref:yang}. Considerando ainda o contexto de análise de imagens, o projeto ImageNet disponibiliza gratuitamente um banco de imagens para aplicações deste domínio e propõe, desde 2010, um desafio anual intitulado \emph{ImageNet Large Scale Visual Recognition Challenge}, que visa elencar os melhores algoritmos em âmbito mundial para detecção de objetos e classificação de imagens em larga escala \cite{ref:image-net}.