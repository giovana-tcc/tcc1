As \textit{Redes Neurais Convolucionais}, do inglês \textit{Convolutional Neural Networks} (CNNs), são modelos de redes neurais especializados em processamento de dados compostos pela união de vários segmentos elementares denominados camadas \cite{ref:goodfellow}. Cada camada possui uma finalidade específica e implementa uma determinada funcionalidade básica como convolução, normalização, \textit{pooling}, etc \cite{ref:khan}.

A camada convolucional é a camada mais importante de uma CNN e utiliza uma operação matemática linear chamada \textit{convolução} \cite{ref:goodfellow}. O processo de convolução é aplicado em um conjunto de \textit{filtros} e uma dada entrada para gerar uma saída conhecida como \textit{mapa de características}. Cada filtro consiste em uma matriz de números discretos que representam os pesos da CNN \cite{ref:khan}.

%cálculo da convolução


A camada convolucional recebe um volume de entrada de largura $w_{in}$, altura $h_{in}$ e profundidade $d_{in}$ e pode possuir um preenchimento  $p$ de zeros (\textit{zero-padding}), aplicado ao redor da entrada. Essa entrada é processada por $k$ filtros que representam os pesos e as conexões da CNN. Cada filtro possui uma extensão espacial $e$, que é igual ao valor da altura e da largura do filtro, e um \textit{stride} $s$, que é a distância entre as aplicações consecutivas do filtro no volume de entrada. A saída da camada de convolução é um volume de largura $w_{out}$ calculado conforme a Equação \ref{eq:wout}, altura $h_{out}$ conforme Equação \ref{eq:hout}  e profundidade $d_{out}$ igual a $k$ \cite{ref:buduma}. 

\begin{equation}
w_{out} = \frac{w_{in} - e + 2p}{s} +1 \label{eq:wout}
\end{equation}

\begin{equation}
h_{out} = \frac{h_{in} - e + 2p}{s} +1 \label{eq:hout}
\end{equation}