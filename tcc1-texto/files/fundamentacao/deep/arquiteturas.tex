\paragraph{LeNet} Yann le Cun desenvolveu, em 1990, um conjunto de redes neurais convolucionais denominado LeNet. Esta arquitetura é composta por duas camadas convolucionais alternadas com camadas de \textit{pooling}. Além destas, as últimas camadas da rede são camadas completamente conectadas baseadas nas tradicionais camadas ocultas da rede MLP \cite{ref:gulli}. 

\paragraph{AlexNet} Em 2012 uma outra arquitetura foi proposta por Alex Krizhevsky, conhecida como AlexNet. A arquitetura AlexNet é mais profunda e uma versão muito mais ampla da arquitetura LeNet \cite{ref:satapathy}. A principal diferença entre a AlexNet e as redes neurais convolucionais predecessoras é a sua maior profundidade que lida muito bem com sua grande quantidade de parâmetros, além da utilização de artifícios como \textit{dropout} e \textit{data augmentation}. As quatro primeiras camadas da arquitetura AlexNet são camadas de convolução e \textit{pooling} de forma similar à LeNet, porém, seguem-se mais três camadas convolucionais. As três últimas camadas são completamente conectadas, mas além destas existem camadas \textit{dropout} que ajudam à reduzir \textit{overfiting}, o que resulta em uma melhor generalização \cite{ref:khan}.

\paragraph{VGGNet} A arquitetura VGGNet é uma das arquiteturas mais populares desde sua criação em 2014. A razão de sua popularidade se dá pela simplicidade do modelo e pelo uso de pequenos ``\textit{kernels}'' de convolução que lidam muito bem com redes profundas. A arquitetura VGGNet usa estritamente \textit{kernels} de convolução de dimensão 3 x 3 combinados com camadas de \textit{pooling} para extração de características e um conjunto de três camadas completamente conectadas para classificação. A utilização dos \textit{kernels} serve para reduzir o número de parâmetros e aumentar a eficiência do treinamento. Além das camadas de convolução, \textit{pooling} e das camadas conectadas, esta arquitetura também possui as camadas \textit{dropout} \cite{ref:khan}.

\paragraph{GoogLeNet} A arquitetura GoogLeNet