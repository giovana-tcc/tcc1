As tecnologias e bibliotecas predominantemente utilizadas neste trabalho envolvem e são compatíveis com a linguagem de programação Python, pois esta tem se destacado amplamente em projetos de Aprendizagem de Máquina em diversos cenários. Esta é uma linguagem de programação  de alto nível, multiparadigma, multiplataforma ...

Para processamentos da imagens em termos de redimensionamento, persistência, mudança e consulta do espaço de cores, as bibliotecas ... tiveram um papel protagonista. No tocante à manipulação de arquivos, contemplando abertura, leitura e busca por extensões similares, as bibliotecas X e Y foram utilizadas. Ademais, no tocante ao treino e teste dos modelos de Aprendizagem de Máquina, as bibliotecas x e y foram consideradas, em que a primeira teve um papel mais predominante no cálculo automático das métricas de desempenho e a segunda nos modelos de DL com parâmetros previamente configurados.

Por fim, a infra-estrutura de computação em nuvem provida pelo Google Cloud ...

\begin{enumerate}
	\item Python
	\begin{itemize}
		\item PIL
		\item numpy
		\item colormath
		\item os
		\item glob
		\item scikit
		\item keras
	\end{itemize}
	\item Gcloud
\end{enumerate}
