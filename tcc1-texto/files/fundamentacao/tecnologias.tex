As tecnologias e bibliotecas predominantemente utilizadas neste trabalho envolvem e são compatíveis com a linguagem de programação Python, pois esta tem se destacado amplamente em projetos de ML em diversos cenários. Esta é uma linguagem de programação interpretada, interativa, multi-paradigma,  de alto nível, multi-plataforma, com uma sintaxe simples e código aberto, idealizada por Guido van Rossum no início da década de 1990 \cite{ref:python}.

Para o processamento da imagens em termos de redimensionamento, persistência, mudança e consulta do espaço de cores, as bibliotecas \emph{Pillow} e \emph{Colormath} tiveram um papel protagonista \cite{lib:pillow,lib:colormath}. No tocante à manipulação de arquivos, contemplando abertura, leitura e busca por extensões similares, as bibliotecas \emph{Os} e \emph{Glob} foram utilizadas \cite{lib:os,lib:glob}. A manipulação do conjunto de imagens e de suas respectivas representações matriciais ficou por conta da \emph{NumPy}, uma biblioteca fundamental para computação científica que é extremamente poderosa para gerenciamento e alterações de matrizes de muitas dimensões \cite{lib:numpy}.  Ademais, no que diz a respeito do treinamento e testes dos modelos de ML, as bibliotecas \emph{Scikit-learn} e \emph{Keras} foram consideradas, em que a primeira teve um papel principal nos cálculos automáticos das métricas de desempenho e a segunda nos modelos de DL com parâmetros previamente configurados \cite{lib:scikit,lib:keras}.

Por fim, a infra-estrutura de computação em nuvem provida pelo Google Cloud Platform (GCP) totalmente voltada para projetos de ML foi essencial para o desenvolvimento deste projeto. As máquinas virtuais oferecidas pela plataforma aumentaram o poder computacional acessível possibilitando um processamento mais favorável ao cenário em questão. A manipulação e pré-processamento do conjunto íntegro de imagens, o treinamento dos modelos de DL e a fase de análise desses modelos foram executados em instâncias disponíveis pelo GCP tendo em vista os recursos de memória e processamento permissíveis \cite{tec:gcloud}.