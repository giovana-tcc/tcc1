\emph{Transfer Learning} (TL), ou Transferência de Conhecimento, é uma poderosa técnica de DL a qual possui diversas aplicações em diferentes domínios \cite{ref:gulli}. Ao invés de estruturar uma arquitetura de uma CNN e treiná-la por completo, esta técnica permite reutilizar uma rede pré-treinada e adaptá-la a um novo conjunto de dados \cite{ref:sewak}. Modelos que foram pré-treinados utilizando um vasto e genérico conjunto de dados conseguem capturar características universais, como por exemplo curvas e arestas, em suas primeiras camadas \cite{ref:zaccone}.

As técnicas de TL podem ser utilizadas de diferentes maneiras, baseando-se nas arquiteturas das CNNs. Existem alguns modelos disponíveis para aplicações que foram pré-treinados utilizando as principais arquiteturas canônicas de CNN e aprenderam as características de grandes conjuntos de dados bastante conhecidos, como o ImageNet e o Places205 \cite{ref:image-net,ref:places205}. Para diferentes tarefas, esses modelos podem ser alterados modificando a camada de saída e fazendo um retreinamento nas últimas camadas das redes para se obter o aprendizado desejado \cite{ref:khan}. 