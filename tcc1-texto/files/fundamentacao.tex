\subsection{Redes Neurais Artificiais}

As \emph{Redes Neurais Artificiais} (RNAs) são modelos computacionais inspirados na capacidade de processamento de informações do cérebro humano \cite{ref2:rojas}.\todo{Sugerir Haykin} De acordo com esta ideia, possuem unidades de processamento simples, denominadas \emph{neurônios artificiais}, dispostos em camadas interconectadas por ligações associadas a coeficientes numéricos, chamados \emph{pesos} []. As RNAs são capazes de aprenderem padrões complexos a partir dos dados e prever resultados para exemplos não conhecidos, o que demonstra a sua capacidade de generalização [].

% Neurônio Artificial

O neurônio artificial é a unidade fundamental na construção de RNAs, tendo sido inspirado no seu análogo biológico. Segundo Rosenblat \todo{citar},  existe um conjunto de $m$ entradas, equivalentes aos dendritos de um neurônio biológico, por onde os sinais são introduzidos. Associa-se um peso a cada entrada, representando a relevância referente a uma conexão sináptica. Há também o peso $w_0$, um termo de polarização criado com a intenção de estabelecer um limitar de ativação para cada neurônio. Este peso corresponde à entrada \emph{bias}, cujo valor é sempre unitário. Pode-se então definir um vetor de entradas $X = [+1, x_1, x_2, \ldots, x_m]$ e um vetor de pesos $W = [w_0, w_1, \ldots, w_m$]. As entradas e pesos são combinados por meio de uma função $\phi: \mathbbm{R}^{m+1} \rightarrow \mathbbm{R}$, que é geralmente a soma ponderada das entradas e pesos, conforme Eq. \ref{eq:somaPonderada}. Este modelo de neurônio encontra-se ilustrado na Figura \ref{img:neuronioArtificial}.

\begin{equation}
\phi(X,W) = \sum_{i =0}^m x_i \cdot w_i \label{eq:somaPonderada}
\end{equation}

\begin{figure}[h!]
	\centering
\includegraphics[width=0.6\textwidth]{./img/neuron}
\caption{Falta legenda!}\label{img:neuronioArtificial}
\end{figure}

A função $f$ é chamada de \emph{função de ativação} e fornece a resposta de um neurônio para uma dada entrada. Esta função é monotônica e contínua, podendo comumente ser as funções identidade, sigmóide hiperbólica, tangente hiperbólica, ou a retificada linear (ReLU). Estas funções encontram-se representadas na Figura X.

\missingfigure{Fazer a figurinha com as quatro funções de ativação.}

\begin{comment}
	\begin{figure}[h!]
		\centering
		\subfloat[Janela de tempo $w = 1$. \label{img:1ano}]{\includegraphics[width=0.5\linewidth]{./img/boxplot1ano}}
		\subfloat[Janela de tempo $w = 2$.\label{img:2anos}]{\includegraphics[width=0.5\linewidth]{./img/boxplot2anos}}\\
		\subfloat[Janela de tempo $w = 3$. \label{img:3anos}]{\includegraphics[width=0.5\linewidth]{./img/boxplot3anos}}
		\subfloat[Janela de tempo $w = 4$.\label{img:4anos}]{\includegraphics[width=0.5\linewidth]{./img/boxplot4anos}}
		\caption{Boxplot da medida \textit{F-score} das 4 técnicas de AM para as quatro janelas de tempo.}
	\end{figure}
\end{comment}

% Multilayer Perceptron
Neurônios artificiais têm uma capacidade computacional limitada, independentemente da função de ativação escolhida. No entanto, um conjunto de neurônios artificiais conectados na forma de uma rede -- \emph{rede neural artificial} -- adquirem a capacidade de resolver problemas de elevada complexidade \cite{Teresa:Livro}.


Disposição dos neurônios em camadas
Papel das camadas ocultas

\missingfigure{Figura com representação de RNA.}


% Paradigma supervisionado
%% Aprendizado sentido forward


% backpropagation




% Exemplos





 As redes do tipo \textit{Multilayer Perceptron} (MLP) pertencem à arquitetura \textit{feedforward} com múltiplas camadas divididas em: camada de entrada, uma ou mais camadas ocultas e camada de saída \cite{ref10:faceli}.

O algoritmo mais tradicional utilizado no processo de aprendizado (treinamento) das redes MLP é o algoritmo de retropropagação do erro ou \textit{backpropagation} \cite{ref11:teive}. Durante o treinamento a rede recebe atributos de entrada que são ponderados e combinados entre as camadas por meio dos neurônios por uma função matemática, chamada função de ativação, gerando ao final um valor de saída. Com base no resultado obtido, a próxima etapa consiste na correção dos pesos de cada neurônio que são ajustados proporcionalmente ao seu erro. Esse processo se repete até que seja alcançado um erro mínimo definido e o treinamento seja interrompido \cite{ref10:faceli,ref1:guedes,ref5:aguni}.

\newpage

As RNAs têm sido utilizadas para aplicações em diversas áreas como Geografia \cite{ref11:teive}, Biologia \cite{ref7:duarte}, Comunicação \cite{ref12:balieiro} e na área Industrial \cite{ref4:prego}. Muitos estudos utilizam as RNAs para classificação de dados, como \cite{ref1:guedes} e \cite{ref3:lima}, ou para previsão de informações como em \cite{ref7:duarte}. No processamento de imagens, as RNAs atuam principalmente em conjunto com as técnicas de Aprendizado Profundo ou \textit{Deep Learnig}.

\begin{itemize}
	\item Ideia
	\item Conceitos
	\begin{itemize}
		\item Camadas -- camada oculta
		\item Neurônios, pesos
		\item Funções de ativação
	\end{itemize}
	\item Aprendizado das RNAs
	\begin{itemize}
		\item Backpropagation
		\item Generalização -- aproximadora universal
	\end{itemize}
	\item Aplicações
\end{itemize}


\subsection{\textit{Deep Learning}}
\emph{Deep Learning} (DL), ou Aprendizado Profundo, é uma subárea do AM especialmente baseada na utilização de RNAs com uma grande quantidade de camadas e neurônios para aprender padrões complexos em um vasto volume de dados \cite{ref:chollet,ref:khan,ref:gulli}. Por meio do reconhecimento de padrões, os modelos baseados em DL são capazes de reconhecer, traduzir, sintetizar e até prever padrões das mais diferentes naturezas \cite{ref:JAI-2017}.

As técnicas de DL têm sido aplicadas com êxito em muitos problemas, especialmente considerando dados de alta dimensionalidade, a exemplo de imagens e vídeos, e contextos em que há uma grande disponibilidade de exemplos  \cite{ref:JAI-2017,ref:khan}. Os modelos de DL têm se destacado, por exemplo, em muitas aplicações de saúde, especialmente considerando a detecção automática de padrões em imagens médicas para fins diagnósticos \cite{ref:yang}. O desafio
\emph{ImageNet Large Scale Visual Recognition Challenge}, de caráter anual realizado desde 2010, também têm promovido a proposição e competição de modelos de vanguarda para fins de detecção de objetos e classificação de imagens em larga escala, contribuindo para o desenvolvimento do estado da arte em VC \cite{ref:image-net}.

Os modelos e técnicas de DL têm sido aplicados em tarefas de aprendizado supervisionado e não supervisionado, em que as redes neurais convolucionais têm sido o modelo mais proeminente. A seção a seguir apresenta o detalhamento deste modelo, suas características e conceitos associados.

\subsubsection{Redes Neurais Convolucionais} \label{subsec:cnn}
As \textit{Redes Neurais Convolucionais}, do inglês \textit{Convolutional Neural Networks} (CNNs), são modelos de redes neurais especializados em processamento de dados compostos pela união de vários segmentos elementares denominados camadas \cite{ref:goodfellow}. Cada camada possui uma finalidade específica e implementa uma determinada funcionalidade básica, como convolução, normalização, \textit{pooling}, etc \cite{ref:khan}.

\subsubsection{Convolução} \label{subsubsec:convolucao}
A \textit{convolução} é uma operação linear que calcula a soma dos produtos de toda a extensão de duas entradas em função de um determinado deslocamento. Essa operação é a propriedade fundamental da camada convolucional, a principal camada de uma CNN \cite{ref:goodfellow}.  O principal objetivo da operação de convolução nas CNN é a extração das características de uma determinada entrada \cite{ref:sewak}. 

O processo de convolução utilizado nas CNNs é aplicado em um conjunto de \textit{filtros} e uma dada entrada para gerar uma saída conhecida como \textit{mapa de características}. A camada convolucional recebe um volume de entrada de $n$ dimensões e pode possuir um preenchimento  $p$ de zeros (\textit{zero-padding}), aplicado ao redor da entrada. Essa entrada é processada por $k$ filtros que representam os pesos e as conexões da CNN \cite{ref:khan}. Cada filtro consiste em uma matriz de números discretos e possui uma extensão espacial $e$, que é igual ao valor da altura e da largura do filtro, e um \textit{stride} $s$, que é a distância entre as aplicações de convolução consecutivas do filtro no volume de entrada \cite{ref:buduma}. 

A Figura \ref{img:convolucao} exemplifica o processo de convolução aplicado em uma entrada de tamanho 5 x 5 e \textit{zero-padding} $p = 1$. O filtro utilizado no exemplo possui extensão espacial $e = 2$ e \textit{stride} $s = 2$ com inicialização de pesos aleatória. É possível observar que o mapa de características é uma saída de tamanho 3 x 3 em que cada um dos seus componentes é a da soma das multiplicações dos elementos do filtro com os elementos de um segmento da entrada \cite{ref:khan}.

\begin{figure}[!ht]
	\centering
	\includegraphics[width=0.9\textwidth]{./img/convolucao}
	\caption{Exemplo de um processo de convolução aplicado em uma entrada de tamanho 5 x 5 e um filtro de tamanho 2 x 2. Fonte: \cite{ref:khan}.}
	\label{img:convolucao}
\end{figure}




\subsubsection{Camadas de uma CNN} \label{subsubsec:camadas}
Para reduzir significativamente a dimensão dos mapas de características, eventualmente, após a camada convolucional, é utilizada uma camada chamada \textit{pooling}. A camada de \textit{pooling} divide o mapa de características de entrada em blocos de tamanhos iguais e processa cada bloco para criar um mapa de características condensado. O processamento dos blocos é definido por uma função de \textit{pool} que pode ser, por exemplo, a função máxima. Assim como na camada de convolução, na camada de \textit{pooling} também é preciso especificar o tamanho da região de \textit{pooling} e o \textit{stride} $s$ da operação, conforme exemplificado na Figura \ref{img:pooling} em que a região possui tamanho 2 x 2 e o \textit{stride} $s=1$ \cite{ref:buduma,ref:khan}.

\begin{figure}[!ht]
	\centering
	\includegraphics[width=0.9\textwidth]{./img/pooling}
	\caption{Exemplo de um processo de \textit{pooling} aplicado em uma entrada de tamanho 5 x 5 e uma região \textit{pooling} de tamanho 2 x 2. Fonte: \cite{ref:khan}.}
	\label{img:pooling}
\end{figure}

\subsubsection{Arquiteturas canônicas de Redes Neurais Convolucionais} \label{subsubsec:arquiteturas}
Como mencionado anteriormente, o desafio anual \emph{ImageNet Large Scale Visual Recognition Challenge} (ILSVRC) têm tido um papel protagonista no desenvolvimento de soluções em DL, pois têm promovido um contexto para proposição e comparação de algumas das arquiteturas de CNNs mais bem sucedidas para problemas de detecção de objetos e classificação de imagens em larga escala \todo{Incluir citação}.

\todo{Uma figura da evolução da competição ano a ano? Quais as métricas? Qual o ganho? Mencionar um pouco mais a contribuição desta competição}

\todo{Como diz no site do ILSVRC: Citation
When reporting results of the challenges or using the datasets, please cite:
Olga Russakovsky*, Jia Deng*, Hao Su, Jonathan Krause, Sanjeev Satheesh, Sean Ma, Zhiheng Huang, Andrej Karpathy, Aditya Khosla, Michael Bernstein, Alexander C. Berg and Li Fei-Fei. (* = equal contribution) ImageNet Large Scale Visual Recognition Challenge. IJCV, 2015. paper | bibtex | paper content on arxiv | attribute annotations. Lá tem o bibtex. Trocar as referências ao longo do texto}

Embora os conceitos das camadas de uma CNN estejam bem estabelecidos e sejam de conhecimento geral, nem sempre é uma tarefa fácil propor uma rede neural deste tipo para um determinado cenário. Assim, uma consequência positiva da realização do ILSVRC é promover a difusão das arquiteturas de destaque na competição, as quais passam ser conhecidas e adaptadas pela comunidade acadêmica e tecnológica na resolução de diversos outros problemas. Considerando esta importância e potencial de aproveitamento de soluções, a seguir são apresentadas algumas destas arquiteturas canônicas.


\paragraph{LeNet} Yann le Cun desenvolveu, em 1990, uma das primeiras arquiteturas utilizadas para o reconhecimento de dígitos manuscritos, a LeNet. Vencedora do ILSVRC 2010, esta arquitetura é composta por três camadas convolucionais alternadas com camadas de \textit{pooling} seguidas de duas FCLs conforme representado na Figura \ref{img:lenet} \cite{ref:sewak,ref:khan}.

\begin{figure}[!ht]
	\centering
	\caption{Arquitetura LeNet de CNN. Fonte: \cite{ref:khan}.}
	\label{img:lenet}
	\includegraphics[width=1\textwidth]{./img/lenet}
\end{figure}


\paragraph{AlexNet} Em 2012, a vencedora do ILSVRC foi a arquitetura proposta por Alex Krizhevsky, conhecida como AlexNet, ilustrada na Figura \ref{img:alexnet}. A AlexNet é mais profunda e uma versão muito mais ampla da arquitetura LeNet \cite{ref:satapathy}. A principal diferença entre a AlexNet e as CNNs predecessoras é a sua maior profundidade, que lida muito bem com sua grande quantidade de parâmetros, além da utilização de artifícios como \textit{dropout} e \textit{data augmentation}. As cinco primeiras camadas da arquitetura AlexNet são camadas de convolução e \textit{pooling} alternadas de forma similar à LeNet, porém, seguem-se mais duas camadas uma convolucional e uma de\textit{pooling}. As três últimas camadas são FCL, mas além destas existem camadas \textit{dropout} que ajudam à reduzir \textit{overfiting} \cite{ref:khan}. \todo{Acho que a citação para alexnet está errada.}

\begin{figure}[!ht]
	\centering
	\caption{Arquitetura da AlexNet. Fonte: \cite{ref:khan}.}
	\label{img:alexnet}
	\includegraphics[width=1\textwidth]{./img/alexnet}
\end{figure}

\paragraph{VGGNet} A arquitetura VGGNet é uma das arquiteturas mais populares desde sua criação em 2014, apesar de não ter sido a vencedora do ILSVRC realizado no respectivo ano. A razão de sua popularidade se dá especialmente em virtude do uso de pequenos filtros de convolução, diminuindo o número de parâmetros ajustáveis e, por conseguinte, aumentando a eficiência do treinamento. A arquitetura VGGNet usa estritamente fitros de convolução de dimensão $3 \times 3$ combinados com camadas de \textit{pooling} para extração de características e um conjunto de três FCLs. Além das camadas de convolução, \textit{pooling} e das camadas conectadas, esta arquitetura também possui as camadas \textit{dropout} como pode ser observado na Figura \ref{img:vggnet} \cite{ref:khan}.

\begin{figure}[!ht]
	\centering
	\caption{Arquitetura VGGNet. Fonte: \cite{ref:khan}.}
	\label{img:vggnet}
	\includegraphics[width=1\textwidth]{./img/vggnet}

\end{figure}

\paragraph{GoogLeNet} Desenvolvida pela empresa Google e vencedora do ILSVRC 2014, a arquitetura GoogLeNet possui $22$ camadas baseadas em um módulo elementar chamado \emph{Inception Module}. O processamento desses módulos ocorre de forma paralela, diferentemente do processamento sequencial das arquiteturas discutidas anteriormente. A ideia central da arquitetura GoogLeNet é paralelizar os módulos e combinar as características da saída sem se preocupar com as funções individuais de cada camada. No entanto, essa abordagem resulta em um mapa de características com muitos elementos, mas para contornar este problema, após a execução do primeiro módulo, a rede realiza uma redução de dimensionalidade utilizando uma FCL antes de continuar o processo de treinamento \cite{ref:khan}. A representação da arquitetura GoogLeNet encontra-se na Figura \ref{img:googlenet}.

\begin{figure}[!ht]
	\centering
	\caption{Arquitetura GoogLeNet. Fonte: \cite{ref:khan}.}
	\label{img:googlenet}
	\includegraphics[width=0.6\textwidth]{./img/googlenet}

\end{figure}



\subsubsection{Arquiteturas Canônicas de Redes Neurais Convolucionais} \label{subsubsec:arquiteturas}
Como mencionado anteriormente, o desafio anual \emph{ImageNet Large Scale Visual Recognition Challenge} (ILSVRC) têm tido um papel protagonista no desenvolvimento de soluções em DL, pois têm promovido um contexto para proposição e comparação de algumas das arquiteturas de CNNs mais bem sucedidas para problemas de detecção de objetos e classificação de imagens em larga escala \todo{Incluir citação}.

\todo{Uma figura da evolução da competição ano a ano? Quais as métricas? Qual o ganho? Mencionar um pouco mais a contribuição desta competição}

\todo{Como diz no site do ILSVRC: Citation
When reporting results of the challenges or using the datasets, please cite:
Olga Russakovsky*, Jia Deng*, Hao Su, Jonathan Krause, Sanjeev Satheesh, Sean Ma, Zhiheng Huang, Andrej Karpathy, Aditya Khosla, Michael Bernstein, Alexander C. Berg and Li Fei-Fei. (* = equal contribution) ImageNet Large Scale Visual Recognition Challenge. IJCV, 2015. paper | bibtex | paper content on arxiv | attribute annotations. Lá tem o bibtex. Trocar as referências ao longo do texto}

Embora os conceitos das camadas de uma CNN estejam bem estabelecidos e sejam de conhecimento geral, nem sempre é uma tarefa fácil propor uma rede neural deste tipo para um determinado cenário. Assim, uma consequência positiva da realização do ILSVRC é promover a difusão das arquiteturas de destaque na competição, as quais passam ser conhecidas e adaptadas pela comunidade acadêmica e tecnológica na resolução de diversos outros problemas. Considerando esta importância e potencial de aproveitamento de soluções, a seguir são apresentadas algumas destas arquiteturas canônicas.


\paragraph{LeNet} Yann le Cun desenvolveu, em 1990, uma das primeiras arquiteturas utilizadas para o reconhecimento de dígitos manuscritos, a LeNet. Vencedora do ILSVRC 2010, esta arquitetura é composta por três camadas convolucionais alternadas com camadas de \textit{pooling} seguidas de duas FCLs conforme representado na Figura \ref{img:lenet} \cite{ref:sewak,ref:khan}.

\begin{figure}[!ht]
	\centering
	\caption{Arquitetura LeNet de CNN. Fonte: \cite{ref:khan}.}
	\label{img:lenet}
	\includegraphics[width=1\textwidth]{./img/lenet}
\end{figure}


\paragraph{AlexNet} Em 2012, a vencedora do ILSVRC foi a arquitetura proposta por Alex Krizhevsky, conhecida como AlexNet, ilustrada na Figura \ref{img:alexnet}. A AlexNet é mais profunda e uma versão muito mais ampla da arquitetura LeNet \cite{ref:satapathy}. A principal diferença entre a AlexNet e as CNNs predecessoras é a sua maior profundidade, que lida muito bem com sua grande quantidade de parâmetros, além da utilização de artifícios como \textit{dropout} e \textit{data augmentation}. As cinco primeiras camadas da arquitetura AlexNet são camadas de convolução e \textit{pooling} alternadas de forma similar à LeNet, porém, seguem-se mais duas camadas uma convolucional e uma de\textit{pooling}. As três últimas camadas são FCL, mas além destas existem camadas \textit{dropout} que ajudam à reduzir \textit{overfiting} \cite{ref:khan}. \todo{Acho que a citação para alexnet está errada.}

\begin{figure}[!ht]
	\centering
	\caption{Arquitetura da AlexNet. Fonte: \cite{ref:khan}.}
	\label{img:alexnet}
	\includegraphics[width=1\textwidth]{./img/alexnet}
\end{figure}

\paragraph{VGGNet} A arquitetura VGGNet é uma das arquiteturas mais populares desde sua criação em 2014, apesar de não ter sido a vencedora do ILSVRC realizado no respectivo ano. A razão de sua popularidade se dá especialmente em virtude do uso de pequenos filtros de convolução, diminuindo o número de parâmetros ajustáveis e, por conseguinte, aumentando a eficiência do treinamento. A arquitetura VGGNet usa estritamente fitros de convolução de dimensão $3 \times 3$ combinados com camadas de \textit{pooling} para extração de características e um conjunto de três FCLs. Além das camadas de convolução, \textit{pooling} e das camadas conectadas, esta arquitetura também possui as camadas \textit{dropout} como pode ser observado na Figura \ref{img:vggnet} \cite{ref:khan}.

\begin{figure}[!ht]
	\centering
	\caption{Arquitetura VGGNet. Fonte: \cite{ref:khan}.}
	\label{img:vggnet}
	\includegraphics[width=1\textwidth]{./img/vggnet}

\end{figure}

\paragraph{GoogLeNet} Desenvolvida pela empresa Google e vencedora do ILSVRC 2014, a arquitetura GoogLeNet possui $22$ camadas baseadas em um módulo elementar chamado \emph{Inception Module}. O processamento desses módulos ocorre de forma paralela, diferentemente do processamento sequencial das arquiteturas discutidas anteriormente. A ideia central da arquitetura GoogLeNet é paralelizar os módulos e combinar as características da saída sem se preocupar com as funções individuais de cada camada. No entanto, essa abordagem resulta em um mapa de características com muitos elementos, mas para contornar este problema, após a execução do primeiro módulo, a rede realiza uma redução de dimensionalidade utilizando uma FCL antes de continuar o processo de treinamento \cite{ref:khan}. A representação da arquitetura GoogLeNet encontra-se na Figura \ref{img:googlenet}.

\begin{figure}[!ht]
	\centering
	\caption{Arquitetura GoogLeNet. Fonte: \cite{ref:khan}.}
	\label{img:googlenet}
	\includegraphics[width=0.6\textwidth]{./img/googlenet}

\end{figure}


\subsubsection{\textit{Transfer Learning}} \label{subsec:transfer}
\emph{Transfer Learning} (TL), ou Transferência de Conhecimento, é uma poderosa técnica de DL a qual possui diversas aplicações em diferentes domínios \cite{ref:gulli}. Ao invés de estruturar uma arquitetura de uma CNN e treiná-la por completo, esta técnica permite reutilizar uma rede pré-treinada e adaptá-la a um novo conjunto de dados \cite{ref:sewak}. Modelos que foram pré-treinados utilizando um vasto e genérico conjunto de dados conseguem capturar características universais, como por exemplo curvas e arestas, em suas primeiras camadas \cite{ref:zaccone}.

As técnicas de TL podem ser utilizadas de diferentes maneiras, baseando-se nas arquiteturas das CNNs. Existem alguns modelos disponíveis para aplicações que foram pré-treinados utilizando as principais arquiteturas canônicas de CNN e aprenderam as características de grandes conjuntos de dados bastante conhecidos, como o ImageNet e o Places205 \cite{ref:image-net,ref:places205}. Para diferentes tarefas, esses modelos podem ser alterados modificando a camada de saída e fazendo um retreinamento nas últimas camadas das redes para se obter o aprendizado desejado \cite{ref:khan}. 


\subsubsection{Redes Neurais Convolucionais}



\paragraph{LeNet} Yann le Cun desenvolveu, em 1990, um conjunto de redes neurais convolucionais denominado LeNet. Esta arquitetura é composta por duas camadas convolucionais alternadas com camadas de \textit{pooling}. Além destas, as últimas camadas da rede são camadas completamente conectadas baseadas nas tradicionais camadas ocultas da rede MLP \cite{ref:gulli}. 

\paragraph{AlexNet} Em 2012 uma outra arquitetura foi proposta por Alex Krizhevsky, conhecida como AlexNet. A arquitetura AlexNet é mais profunda e uma versão muito mais ampla da arquitetura LeNet \cite{ref:satapathy}. A principal diferença entre a AlexNet e as redes neurais convolucionais predecessoras é a sua maior profundidade que lida muito bem com sua grande quantidade de parâmetros, além da utilização de artifícios como \textit{dropout} e \textit{data augmentation}. As quatro primeiras camadas da arquitetura AlexNet são camadas de convolução e \textit{pooling} de forma similar à LeNet, porém, seguem-se mais três camadas convolucionais. As três últimas camadas são completamente conectadas, mas além destas existem camadas \textit{dropout} que ajudam à reduzir \textit{overfiting}, o que resulta em uma melhor generalização \cite{ref:khan}.

\paragraph{VGGNet} A arquitetura VGGNet é uma das arquiteturas mais populares desde sua criação em 2014. A razão de sua popularidade se dá pela simplicidade do modelo e pelo uso de pequenos ``\textit{kernels}'' de convolução que lidam muito bem com redes profundas. A arquitetura VGGNet usa estritamente \textit{kernels} de convolução de dimensão 3 x 3 combinados com camadas de \textit{pooling} para extração de características e um conjunto de três camadas completamente conectadas para classificação. A utilização dos \textit{kernels} serve para reduzir o número de parâmetros e aumentar a eficiência do treinamento. Além das camadas de convolução, \textit{pooling} e das camadas conectadas, esta arquitetura também possui as camadas \textit{dropout} \cite{ref:khan}.

\paragraph{GoogLeNet} A arquitetura GoogLeNet 

