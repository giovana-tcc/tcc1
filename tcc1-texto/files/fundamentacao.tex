\subsection{Redes Neurais Artificiais}
Redes Neurais Artificiais (RNAs) são modelos computacionais inspirados na capacidade de processamento de informações do cérebro humano [1,2]. Estas redes utilizam  métodos matemáticos que permitem o aprendizado de sistemas complexos podendo realizar qualquer tipo de mapeamento linear ou não-linear [6,7]. Estes modelos são capazes de analisar informações sem nenhum conhecimento prévio sobre a sua distribuição e, portanto, são bastante apropriados para o reconhecimento de padrões e para a classificação de dados devido à sua capacidade de generalização [4,8].

As RNAs são compostas por elementos básicos de processamento, conhecidos como neurônios artificiais, dispostos em camadas interconectadas por ligações associadas a coeficientes numéricos (pesos) [4,9]. As redes do tipo \textit{Multilayer Perceptron} (MLP) pertencem à arquitetura \textit{feedforward} com múltiplas camadas divididas em: camada de entrada, uma ou mais camadas ocultas e camada de saída [10]. 

O algoritmo mais tradicional utilizado no processo de aprendizado (treinamento) das redes MLP é o algoritmo de retropropagação do erro ou \textit{backpropagation} [11]. Durante o treinamento a rede recebe atributos de entrada que são ponderados e combinados entre as camadas por meio dos neurônios por uma função matemática, chamada função de ativação, gerando ao final um valor de saída. Com base no resultado obtido, a próxima etapa consiste na correção dos pesos de cada neurônio que são ajustados proporcionalmente ao seu erro. Esse processo se repete até que seja alcançado um erro mínimo definido e o treinamento seja interrompido [1,10,5].

As RNAs têm sido utilizadas para aplicações em diversas áreas como Geografia [11], Biologia [7], Comunicação [12] e na área Industrial [4]. Muitos estudos utilizam as RNAs para classificação de dados, como [1] e [3], ou para previsão de informações como em [7]. No processamento de imagens, as RNAs atuam principalmente em conjunto com as técnicas de Aprendizado Profundo ou \textit{Deep Learnig}. 

\begin{itemize}
	\item Ideia
	\item Conceitos
	\begin{itemize}
		\item Camadas -- camada oculta
		\item Neurônios, pesos
		\item Funções de ativação
	\end{itemize}
	\item Aprendizado das RNAs
	\begin{itemize}
		\item Backpropagation
		\item Generalização -- aproximadora universal
	\end{itemize}
	\item Aplicações
\end{itemize}
\subsection{\textit{Deep Learning}}
\subsubsection{Redes Neurais Convolucionais}