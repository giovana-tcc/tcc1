% O começo da introdução vai aqui
As técnicas de Aprendizado de Máquina (AM) têm sido aplicadas com sucesso em um grande número de problemas reais em diversos domínios. A principal razão deste sucesso é decorrente da natureza inferencial e da boa capacidade de generalização dos métodos e técnicas desta área, cuja ideia central consiste em utilizar algoritmos capazes de aprender padrões por meio de exemplos, baseando-se apenas em dados previamente disponíveis \cite{ref:faceli,ref:gulli}.

A Visão Computacional (VC), por sua vez, é uma área que procura desenvolver métodos capazes de replicar nos computadores as capacidades da visão humana. O reconhecimento de imagens é uma parte integrante do processo de VC e aplica as técnicas de processamento digital de imagens para extrair as características desejadas. Antes da difusão das técnicas de AM, o procedimento de extração de características de imagens envolvia um grande esforço devido ao processamento e à aplicação de uma variedade de métodos matemáticos. Em algumas situações, por exemplo, até mesmo análises de especialistas eram utilizadas para descobrir as regras necessárias para o reconhecimento e extração de certos padrões \cite{ref:faceli}. \todo{Referência de VC. Gonzalez}

Devido ao progresso significativo no campo da VC e da tecnologia de sensores visuais, VC e  AM desempenharam juntos papéis decisivos no desenvolvimento de uma variedade de aplicações baseadas em processamento digital de imagens \cite{ref:khan}. Por exemplo, o reconhecimento de dígitos manuscritos proposto pelo \textit{dataset} MNIST (...sigla) foi uma das aplicações pioneiras que congregou VC e AM, difundida até os dias atuais para fins didáticos de métodos e técnicas deste domínio \todo{Citar o MNIST}. Atualmente, em particular, um dos projetos emergentes é o ImageNet, o qual disponibiliza um grande banco de imagens projetado para tarefas de VC e propõe anualmente um desafio chamado \emph{ImageNet Large Scale Visual Recognition Challenge}, que visa elencar os melhores algoritmos em âmbito mundial para detecção de objetos e classificação de imagens em larga escala \todo{citar site do Imagenet}.

Nas últimas décadas, com a crescente complexidade dos problemas a serem tratados computacionalmente e diante do grande volume de dados constantemente gerados por diferentes setores, tornou-se clara a necessidade de ferramentas, algoritmos, métodos e técnicas computacionais mais sofisticados para endereçar estas questões \cite{ref:faceli}. Embora alguns algoritmos de AM já existissem há bastante tempo, a demanda por aplicar automaticamente cálculos matemáticos complexos em uma grande escala de dados tornou-se uma necessidade particularmente mais recente. Embora tenha havido um aumento do poder de processamento e armazenamento pelos dispositivos computacionais disponíveis, especialmente via Computação em Nuvem, novas estratégias de AM precisavam ser desenvolvidas ou evoluírem a partir de técnicas já existentes. Neste sentido, as técnicas de \emph{Deep Learning} (DL) emergiram, principalmente compreendendo as redes neurais com uma grande quantidade de camadas ocultas e operações convolucionais \cite{ref:khan}.

% Parágrafo conceituando deep learning
Entende-se por DL, o conjunto de ... cuja principal característica é .... (definição para DL) [citação].  As três principais vantagens oferecidas pelo DL são a simplicidade da construção das redes, a escalabilidade, por lidar com um grande volume de dados, e a transferência de conhecimento (do inglês, \textit{Transfer Learning} -- TL), pois um modelo treinado para uma determinada tarefa pode ser adaptado para outras tarefas relacionadas \cite{ref:khan}.

Considerando este cenário e almejando a aplicação destes conceitos de vanguarda, a proposta do presente trabalho de conclusão de curso visa explorar as arquiteturas canônicas de redes neurais convolucionais, utilizando as técnicas de DL e TL, para endereçar o problema da colorização artificial de imagens. Neste problema, uma imagem em tons de cinza, a exemplo de uma imagem histórica, é apresentada à uma rede neural convolucional que, como resposta, propõe uma versão colorida da mesma. Essa colorização deve ser plausível e realista ao ponto de não possuir inconsistência visual quando analisada por pessoas comuns.

Ao longo desta introdução serão mostrados os demais elementos que compõem este trabalho. A Seção \ref{subsec:objetivos} contempla os objetivos propostos para o desenvolvimento do projeto. Na Seção \ref{subsec:jutificativa} são apresentadas as justificativas  que motivam a realização do trabalho em questão. A metodologia adotada é detalhada na Seção \ref{subsec:metodologia}. Por fim, a Seção \ref{subsec:cronograma}  compreende o cronograma das atividades, seguido da Seção \ref{subsec:organizacao} que dispõe a organização do restante do documento.

\subsection{Objetivos} \label{subsec:objetivos}
O objetivo geral deste trabalho consiste em explorar estratégias para colorização artificial de imagens utilizando técnicas de \textit{Deep Learning}. Para tanto, faz-se necessário elencar alguns objetivos específicos, descritos a seguir:

\begin{enumerate}
	\item Consolidar uma base de dados representativa de imagens coloridas para treinamento das redes;
	\item Descrever o problema da colorização artificial de imagens segundo uma tarefa de Aprendizado de Máquina;
	\item Explorar a utilização das arquiteturas canônicas de redes neurais convolucionais mediante \textit{Transfer Learning} aplicadas ao problema  considerado;
	\item Propor, treinar e testar diferentes redes neurais convolucionais baseadas nas arquiteturas canônicas elencadas;
	\item Analisar os resultados obtidos de maneira quantitativa e qualitativa.
\end{enumerate}


\subsection{Justificativa} \label{subsec:jutificativa}
Imagens em escala de cinzas retêm informações que podem ser importantes em diversos aspectos. A colorização de fotos em arquivos antigos pode agregar algum valor aos seus respectivos contextos históricos e artísticos. Algum detalhe de uma imagem em tons de cinza talvez possua outra interpretação se esta mesma imagem estivesse colorizada. Seguindo o mesmo raciocínio, a coloração das imagens de câmeras de segurança com baixa resolução pode influenciar nas interpretações das filmagens. 

Na área da saúde, colorizações podem ser ajustadas e modificadas para restaurar algum tipo de perturbação visual. Utilizando as técnicas de colorização com \textit{Deep Learning}, algumas cores podem ser corrigidas para melhorar a visualização de, por exemplo, portadores de daltonismo.

Além disso, a proposta deste trabalho também incentiva a prática de conceitos, técnicas e tecnologias de uma área emergente da computação, contribuindo na formação profissional da aluna concluinte. \todo{Incluir parágrafo default sobre o LSI}.

\subsection{Metodologia} \label{subsec:metodologia}
Para atingir os objetivos propostos no escopo deste trabalho, a condu��o das atividades obedece � metodologia apresentada a seguir, composta dos seguintes passos:

\begin{enumerate}
	\item Estudo dos conceitos relacionados � Aprendizado de M�quina, \textit{Deep Learning} e as principais arquiteturas de redes neurais convolucionais;
	\item Estudo do ferramental tecnol�gico para elabora��o e execu��o de projetos de \textit{Deep Learning}, incluindo Python, Keras, Sci-kit Learn, Google Cloud Platform, dentre outros;
	\item Elaborar uma base de dados representativa de imagens coloridas e em escalas de cinza para fins de aprendizado dos padr�es de colora��o pelas redes neurais convolucionais;
	\item Elencar um conjunto de arquiteturas can�nicas das redes neurais convolucionais aplic�veis ao problema em quest�o;
	\item Propor modifica��es nas redes neurais identificadas no passo anterior mediante \textit{Transfer Learning};
	\item Treinar as redes modificadas com os exemplos da base de dados;
	\item Testar as redes e coletar m�tricas de desempenho;
	\item Analisar os resultados obtidos identificando as redes mais adequadas ao cen�rio considerado;
	\item Escrita da proposta do Trabalho de Conclus�o de Curso;
	\item Defesa da proposta do Trabalho de Conclus�o de Curso;
	\item Escrita do Trabalho de Conclus�o de Curso;
	\item Defesa do Trabalho de Conclus�o de Curso.
\end{enumerate}


\subsection{Cronograma} \label{subsec:cronograma}
\begin{frame}{Cronograma}
\begin{table}[!ht]
	\centering
	\tiny
	\begin{tabular}{p{3.4cm}lp{0.1cm} c p{0.1cm}cp{0.1cm} c p{0.1cm}cp{0.1cm} c p{0.1cm}cp{0.1cm} c p{0.1cm}cp{0.1cm} c p{0.1cm}cp{0.1cm} c }
			\toprule
	& & & & & & \textbf{2018} & & & & & \Tstrut \\
	& \textbf{02} & \textbf{03} & \textbf{04} & \textbf{05} & \textbf{06} & \textbf{07} & \textbf{08} & \textbf{09} & \textbf{10} & \textbf{11} & \textbf{12}  \Tstrut \\
	\midrule
	\textbf{Estudo dos conceitos te�ricos relacionados � \textit{Machine Learning}} & x & x & x & x & & & & & & & \Tstrut \\
	\textbf{Estudo do ferramental tecnol�gico para elabora��o do projeto} & x & x & x & x & & & & & & & \Tstrut \\
	\textbf{Consolida��o da base de dados} & & & x & x & & & & & & & \Tstrut \\
	\textbf{Especifica��o das arquiteturas can�nicas de redes neurais convolucionais} & & & x & & & & & & & & \Tstrut \\
	\textbf{Aplica��o das t�cnidas de \textit{Transfer Learning} nas redes neurais convolucionais identificadas} & & & & x & & & & & & & \Tstrut \\
	\textbf{Treinamento das redes neurais convolucionais com os exemplos da base de dados} & & & & x & x & x & x & & & & \Tstrut \\
	\textbf{Testes das redes e compara��o de m�tricas de desempenho} & & & & & x & x & x & x & & & \Tstrut \\
	\textbf{An�lise dos resultados obtidos} & & & & & & & & & x & x & x \Tstrut \\
	\textbf{Escrita da proposta do trabalho} & x & x & x & x & x & & & & & & \Tstrut \\
	\textbf{Defesa da proposta do trabalho} & & & & & x & & & & & & \Tstrut \\
	\textbf{Escrita do trabalho final} & & & & & & x & x & x & x & x & x \Tstrut \\
	\textbf{Defesa do trabalho final} & & & & & & & & & & & x\Tstrut \\
	\bottomrule
	\end{tabular}
	
	\label{tab:cronograma}
\end{table}
\end{frame}

\subsection{Organização do Documento} \label{subsec:organizacao}
Para apresentar a proposta do presente trabalho de conclusão de curso, este documento está organizado como segue. A Seção \ref{sec:fundamentacao} contempla os fundamentos teóricos necessários para elaboração do projeto, incluindo conceitos de Aprendizado de Máquina, \textit{Deep Learning} e as principais arquiteturas de redes neurais convolucionais.  A solução proposta para o tema de colorização de imagens utilizando técnicas de \textit{Deep Learning} é abordada na Seção \ref{sec:solucao}. A Seção \ref{sec:relacionados} discorre sobre os trabalhos relacionados. Por fim, as considerações parciais do trabalho encontram-se na Seção \ref{sec:consideracoes}.
