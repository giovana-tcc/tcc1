Para atingir os objetivos propostos no escopo deste trabalho, a condução das atividades obedece à metodologia apresentada a seguir, composta dos seguintes passos:

\begin{enumerate}
	\item Estudo dos conceitos relacionados à Aprendizado de Máquina, \textit{Deep Learning} e as principais arquiteturas de redes neurais convolucionais;
	\item Estudo do ferramental tecnológico para elaboração e execução de projetos de \textit{Deep Learning}, incluindo Python, Keras, Sci-kit Learn, Google Cloud Platform, dentre outros;
	\item Elaborar uma base de dados representativa de imagens coloridas e em escalas de cinza para fins de aprendizado dos padrões de coloração pelas redes neurais convolucionais;
	\item Elencar um conjunto de arquiteturas canônicas das redes neurais convolucionais aplicáveis ao problema em questão;
	\item Propor modificações nas redes neurais identificadas no passo anterior mediante \textit{Transfer Learning};
	\item Treinar as redes modificadas com os exemplos da base de dados;
	\item Testar as redes e coletar métricas de desempenho;
	\item Analisar os resultados obtidos identificando as redes mais adequadas ao cenário considerado;
	\item Escrita da proposta do Trabalho de Conclusão de Curso;
	\item Defesa da proposta do Trabalho de Conclusão de Curso;
	\item Escrita do Trabalho de Conclusão de Curso;
	\item Defesa do Trabalho de Conclusão de Curso.
\end{enumerate}
