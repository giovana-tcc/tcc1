Imagens em escala de cinzas retêm informações que podem ser importantes em diversos aspectos. A colorização de fotos em arquivos antigos pode agregar algum valor aos seus respectivos contextos históricos e artísticos. Algum detalhe de uma imagem em tons de cinza talvez possua outra interpretação se esta mesma imagem estivesse colorizada. Seguindo o mesmo raciocínio, a coloração das imagens de câmeras de segurança com baixa resolução pode influenciar nas interpretações das filmagens. 

Na área da saúde, colorizações podem ser ajustadas e modificadas para restaurar algum tipo de perturbação visual. Utilizando as técnicas de colorização com \textit{Deep Learning}, algumas cores podem ser corrigidas para melhorar a visualização de, por exemplo, portadores de daltonismo.

Além disso, a proposta deste trabalho também incentiva a prática de conceitos, técnicas e tecnologias de uma área emergente da computação, contribuindo na formação profissional da aluna concluinte. \todo{Incluir parágrafo default sobre o LSI}.