Imagens em escala de cinzas retêm informações que podem ser importantes em diversos aspectos. A colorização de fotos em arquivos antigos pode agregar algum valor aos seus respectivos contextos históricos e artísticos. Algum detalhe de uma imagem em tons de cinza talvez possua outra interpretação se esta mesma imagem estivesse colorizada. Seguindo o mesmo raciocínio, a coloração das imagens de câmeras de segurança com baixa resolução pode influenciar nas interpretações das filmagens, permitindo, por exemplo, a identificação mais apropriada de indivíduos presentes nestas imagens.s

Na área da Saúde, por exemplo, colorizações podem ser ajustadas e modificadas para restaurar algum tipo de perturbação visual de indivíduos, como é o caso do daltonismo. As técnicas de colorização artificial a serem exploradas neste trabalho podem, por exemplo, possibilitar representações visuais mais adequadas para os portadores deste tipo de agravo.

Além do que foi exposto, a proposta considerada neste trabalho incentiva a prática de conceitos, técnicas e tecnologias de uma área emergente da Computação, contribuindo na formação profissional da aluna concluinte. No mais, esta proposta de trabalho de conclusão de curso está alinhada com as atividades desenvolvidas pelo \emph{Laboratório de Sistemas Inteligentes} (LSI), uma iniciativa promovida por docentes do Núcleo de Computação (NUCOMP) da Escola Superior de Tecnologia (EST) da Universidade do Estado do Amazonas (UEA), motivando o desenvolvimento de soluções inovadoras que utilizam técnicas emergentes do Aprendizado de Máquina.
